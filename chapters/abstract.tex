%!TEX root = ../report.tex
\documentclass[../report.tex]{subfiles}
\begin{document}
    \begin{abstract}
    	\noindent
        GestaltMatcher is a state-of-the-art convolutional neural network (CNN) based tool developed to
        identify rare genetic syndromes from frontal facial images of people. It surpasses the ability of clinicians to recognize images of certain rare genetic conditions. However, the black-box nature of the CNN model limits the scientific community from finding the facial regions it focuses on to make predictions. Such knowledge may help the community to improve their understanding of facial feature - genetic syndrome associations. Besides, it makes GestaltMatcher more transparent and dependable for clinical practitioners who use the tool to assist with their diagnoses.
        
        This research work attempts to determine the CNN’s attention regions using feature attribution methods
        from the realm of explainable artificial intelligence (XAI). A literature review is conducted to identify 
        state-of-the-art attribution methods suitable for the model and data at hand. Subsequently, the identified
        methods are tweaked and applied to the CNN model to generate explanations for its predictions in the form of attribution maps. Explanation artifacts are generated for instances from 139 syndrome classes
        in the GestaltMatcher database dataset, a curation of facial images of patients with rare genetic
        conditions. In addition, characteristic representations for facial images and attribution maps of each
        category are obtained, in the forms of the composite face and syndrome-wise attribution maps, respectively.
        Finally, the set of three experimental artifacts representing Cornelia de Lange syndrome (CDLS), Williams
        Beuren syndrome (WBS) and Hyperphosphatasia with mental retardation syndrome (HPMRS) are
        evaluated by an experienced clinical geneticist.
        
        Evaluation results show that the regions highlighted in attribution maps generated using the full-gradient
        representation (FullGrad) method closely matched the clinician’s attention regions in six out of nine
        samples from CDLS and HPMRS classes. In addition, the clinician confirmed the presence of at least a
        facial feature of CDLS, WBS, and HPMRS in their respective characteristic representations.
        Analyzing attribution maps of other classes reveals that the GestaltMatcher model predominantly considers
        features of the nasal region to recognize syndromes.    
    \end{abstract}
\end{document}
